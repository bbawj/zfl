\documentclass[12pt]{article}
\title{Emulation Framework for AIoT Federated Learning}
\author{Brendan Ang Wei Jie}
\begin{document}

\maketitle

\pagebreak
\section{Abstract}
Federated learning allows a fleet of devices to collaborate towards a globally trained machine
learning model. Research has continued to produce novel federated learning algorithms to tackle
different issues in FL such as heterogeneity and learning over data from non-identical
distributions. Performance of these algorithms depend in part on the system parameters used in FL
such as number of clients and number of passes. Furthermore, a realistic benchmark would require
one to procure a large fleet of devices. This work seeks to introduce a software emulation framework to
streamline the process of building a fleet of clients and allow easy testing of FL system
parameters for configuration of optimal values.
\pagebreak
  \section{Acknowledgements}
\pagebreak
\tableofcontents
\pagebreak
\section{Introduction}
Federated learning emerged as a method for solving key issues with the standard centralized learning
approach. Some of these issues are (1) Preserving user data privacy: a centralized training approach involves the need for the central
machine performing the computation to have full access to all the data. With FL, data never
leaves each individual client's device. Instead, only the updated weights are shared to form the
global model.(2) Scalability: FL enables leveraging a network to perform computation in parallel.

However, with the distributed nature of FL comes a few key challenges.
\begin{itemize}
  \item Limited computing resources: individual devices may have constraints on memory, limiting the
    size of the local model it is able to train. Constraints on computing power can also lead to
    longer training times. This is particularly so for Internet of Things (IoT) devices such as
    sensors, where their embedded nature leads to a limited size and power.
  \item Network limitations: communication speed can become the bottleneck for performance as IoT
    devices rely on unstable wireless communication networks. Furthermore, data constraints can
    exist such that the number of bytes sent may be a cost inducing factor.
\end{itemize}
\subsection{Problem}
However, prior to effectively deploying FL on resource-constrained mobile devices in large scale, different factors including the convergence rate, energy efficiency and model accuracy should be well studied.

FLSim\cite{} aims to provide a simulation framework for FL by offering users a set of software components
which can then be mixed and matched to create a simulator for their use case. Developers need only
define the data, model and metrics reported, and FL system parameters can be altered in a JSON
configuration file. FLUTE\cite{} adds additional features by allowing users to gain access to cloud based
compute and data. However, these simulation solutions are consequently hardware-agnostic. Without
taking into account the specific platforms which FL is performed on, these simulators cannot provide
insight into whether the system will work in the production environment. Here, this work proposes
a new emulation framework zfl, which aims to solve this issue by running FL on emulated hardware.
\subsection{QEMU}
QEMU\cite{} is an open source machine emulator and virtualizer. It enables system emulation, where it
provides a virtual model of an entire machine (CPU, memory and emulated devices) to run a guest OS.
In this mode the CPU may be fully emulated, or it may work with a hypervisor to allow the guest to
run directly on the host CPU. In zfl, full CPU emulation is used without a hypervisor.
\subsection{Zephyr OS}
To emulate the issue of limited computing resources, it is important to make use of system runtimes
used by those devices. In particular, the type of operating system used will help to ensure that the kernel is
lightweight and configurable.
Zephyr OS\cite{} is one such OS. It is based on a small-footprint kernel designed for use on resource-constrained and embedded systems: from simple embedded environmental sensors and LED wearables to sophisticated embedded controllers, smart watches, and IoT wireless applications.
Furthermore, Zephyr is highly configurable, allowing the user to choose only the specific kernel
services required, and also delve into lower level memory allocations of the system RAM. In zfl,
client code will be written to work in the Zephyr OS environment.
\section{Implementation}
\section{Experiments}
\section{Limitations}
\end{document}
